%% bare_jrnl.tex
%% V1.4b
%% 2015/08/26
%% by Michael Shell
%% see http://www.michaelshell.org/
%% for current contact information.
%%
%% This is a skeleton file demonstrating the use of IEEEtran.cls
%% (requires IEEEtran.cls version 1.8b or later) with an IEEE
%% journal paper.
%%
%% Support sites:
%% http://www.michaelshell.org/tex/ieeetran/
%% http://www.ctan.org/pkg/ieeetran
%% and
%% http://www.ieee.org/

%%*************************************************************************
%% Legal Notice:
%% This code is offered as-is without any warranty either expressed or
%% implied; without even the implied warranty of MERCHANTABILITY or
%% FITNESS FOR A PARTICULAR PURPOSE! 
%% User assumes all risk.
%% In no event shall the IEEE or any contributor to this code be liable for
%% any damages or losses, including, but not limited to, incidental,
%% consequential, or any other damages, resulting from the use or misuse
%% of any information contained here.
%%
%% All comments are the opinions of their respective authors and are not
%% necessarily endorsed by the IEEE.
%%
%% This work is distributed under the LaTeX Project Public License (LPPL)
%% ( http://www.latex-project.org/ ) version 1.3, and may be freely used,
%% distributed and modified. A copy of the LPPL, version 1.3, is included
%% in the base LaTeX documentation of all distributions of LaTeX released
%% 2003/12/01 or later.
%% Retain all contribution notices and credits.
%% ** Modified files should be clearly indicated as such, including  **
%% ** renaming them and changing author support contact information. **
%%*************************************************************************


% *** Authors should verify (and, if needed, correct) their LaTeX system  ***
% *** with the testflow diagnostic prior to trusting their LaTeX platform ***
% *** with production work. The IEEE's font choices and paper sizes can   ***
% *** trigger bugs that do not appear when using other class files.       ***                          ***
% The testflow support page is at:
% http://www.michaelshell.org/tex/testflow/



\documentclass[journal]{IEEEtran}
%
% If IEEEtran.cls has not been installed into the LaTeX system files,
% manually specify the path to it like:
% \documentclass[journal]{../sty/IEEEtran}





% Some very useful LaTeX packages include:
% (uncomment the ones you want to load)


% *** MISC UTILITY PACKAGES ***
%
%\usepackage{ifpdf}
% Heiko Oberdiek's ifpdf.sty is very useful if you need conditional
% compilation based on whether the output is pdf or dvi.
% usage:
% \ifpdf
%   % pdf code
% \else
%   % dvi code
% \fi
% The latest version of ifpdf.sty can be obtained from:
% http://www.ctan.org/pkg/ifpdf
% Also, note that IEEEtran.cls V1.7 and later provides a builtin
% \ifCLASSINFOpdf conditional that works the same way.
% When switching from latex to pdflatex and vice-versa, the compiler may
% have to be run twice to clear warning/error messages.



% *** CITATION PACKAGES ***
%
%\usepackage{cite}
% cite.sty was written by Donald Arseneau
% V1.6 and later of IEEEtran pre-defines the format of the cite.sty package
% \cite{} output to follow that of the IEEE. Loading the cite package will
% result in citation numbers being automatically sorted and properly
% "compressed/ranged". e.g., [1], [9], [2], [7], [5], [6] without using
% cite.sty will become [1], [2], [5]--[7], [9] using cite.sty. cite.sty's
% \cite will automatically add leading space, if needed. Use cite.sty's
% noadjust option (cite.sty V3.8 and later) if you want to turn this off
% such as if a citation ever needs to be enclosed in parenthesis.
% cite.sty is already installed on most LaTeX systems. Be sure and use
% version 5.0 (2009-03-20) and later if using hyperref.sty.
% The latest version can be obtained at:
% http://www.ctan.org/pkg/cite
% The documentation is contained in the cite.sty file itself.






% *** GRAPHICS RELATED PACKAGES ***
%
\ifCLASSINFOpdf
  % \usepackage[pdftex]{graphicx}
  % declare the path(s) where your graphic files are
  % \graphicspath{{../pdf/}{../jpeg/}}
  % and their extensions so you won't have to specify these with
  % every instance of \includegraphics
  % \DeclareGraphicsExtensions{.pdf,.jpeg,.png}
\else
  % or other class option (dvipsone, dvipdf, if not using dvips). graphicx
  % will default to the driver specified in the system graphics.cfg if no
  % driver is specified.
  % \usepackage[dvips]{graphicx}
  % declare the path(s) where your graphic files are
  % \graphicspath{{../eps/}}
  % and their extensions so you won't have to specify these with
  % every instance of \includegraphics
  % \DeclareGraphicsExtensions{.eps}
\fi
% graphicx was written by David Carlisle and Sebastian Rahtz. It is
% required if you want graphics, photos, etc. graphicx.sty is already
% installed on most LaTeX systems. The latest version and documentation
% can be obtained at: 
% http://www.ctan.org/pkg/graphicx
% Another good source of documentation is "Using Imported Graphics in
% LaTeX2e" by Keith Reckdahl which can be found at:
% http://www.ctan.org/pkg/epslatex
%
% latex, and pdflatex in dvi mode, support graphics in encapsulated
% postscript (.eps) format. pdflatex in pdf mode supports graphics
% in .pdf, .jpeg, .png and .mps (metapost) formats. Users should ensure
% that all non-photo figures use a vector format (.eps, .pdf, .mps) and
% not a bitmapped formats (.jpeg, .png). The IEEE frowns on bitmapped formats
% which can result in "jaggedy"/blurry rendering of lines and letters as
% well as large increases in file sizes.
%
% You can find documentation about the pdfTeX application at:
% http://www.tug.org/applications/pdftex





% *** MATH PACKAGES ***
%
%\usepackage{amsmath}
% A popular package from the American Mathematical Society that provides
% many useful and powerful commands for dealing with mathematics.
%
% Note that the amsmath package sets \interdisplaylinepenalty to 10000
% thus preventing page breaks from occurring within multiline equations. Use:
%\interdisplaylinepenalty=2500
% after loading amsmath to restore such page breaks as IEEEtran.cls normally
% does. amsmath.sty is already installed on most LaTeX systems. The latest
% version and documentation can be obtained at:
% http://www.ctan.org/pkg/amsmath





% *** SPECIALIZED LIST PACKAGES ***
%
%\usepackage{algorithmic}
% algorithmic.sty was written by Peter Williams and Rogerio Brito.
% This package provides an algorithmic environment fo describing algorithms.
% You can use the algorithmic environment in-text or within a figure
% environment to provide for a floating algorithm. Do NOT use the algorithm
% floating environment provided by algorithm.sty (by the same authors) or
% algorithm2e.sty (by Christophe Fiorio) as the IEEE does not use dedicated
% algorithm float types and packages that provide these will not provide
% correct IEEE style captions. The latest version and documentation of
% algorithmic.sty can be obtained at:
% http://www.ctan.org/pkg/algorithms
% Also of interest may be the (relatively newer and more customizable)
% algorithmicx.sty package by Szasz Janos:
% http://www.ctan.org/pkg/algorithmicx




% *** ALIGNMENT PACKAGES ***
%
%\usepackage{array}
% Frank Mittelbach's and David Carlisle's array.sty patches and improves
% the standard LaTeX2e array and tabular environments to provide better
% appearance and additional user controls. As the default LaTeX2e table
% generation code is lacking to the point of almost being broken with
% respect to the quality of the end results, all users are strongly
% advised to use an enhanced (at the very least that provided by array.sty)
% set of table tools. array.sty is already installed on most systems. The
% latest version and documentation can be obtained at:
% http://www.ctan.org/pkg/array


% IEEEtran contains the IEEEeqnarray family of commands that can be used to
% generate multiline equations as well as matrices, tables, etc., of high
% quality.




% *** SUBFIGURE PACKAGES ***
%\ifCLASSOPTIONcompsoc
%  \usepackage[caption=false,font=normalsize,labelfont=sf,textfont=sf]{subfig}
%\else
%  \usepackage[caption=false,font=footnotesize]{subfig}
%\fi
% subfig.sty, written by Steven Douglas Cochran, is the modern replacement
% for subfigure.sty, the latter of which is no longer maintained and is
% incompatible with some LaTeX packages including fixltx2e. However,
% subfig.sty requires and automatically loads Axel Sommerfeldt's caption.sty
% which will override IEEEtran.cls' handling of captions and this will result
% in non-IEEE style figure/table captions. To prevent this problem, be sure
% and invoke subfig.sty's "caption=false" package option (available since
% subfig.sty version 1.3, 2005/06/28) as this is will preserve IEEEtran.cls
% handling of captions.
% Note that the Computer Society format requires a larger sans serif font
% than the serif footnote size font used in traditional IEEE formatting
% and thus the need to invoke different subfig.sty package options depending
% on whether compsoc mode has been enabled.
%
% The latest version and documentation of subfig.sty can be obtained at:
% http://www.ctan.org/pkg/subfig




% *** FLOAT PACKAGES ***
%
%\usepackage{fixltx2e}
% fixltx2e, the successor to the earlier fix2col.sty, was written by
% Frank Mittelbach and David Carlisle. This package corrects a few problems
% in the LaTeX2e kernel, the most notable of which is that in current
% LaTeX2e releases, the ordering of single and double column floats is not
% guaranteed to be preserved. Thus, an unpatched LaTeX2e can allow a
% single column figure to be placed prior to an earlier double column
% figure.
% Be aware that LaTeX2e kernels dated 2015 and later have fixltx2e.sty's
% corrections already built into the system in which case a warning will
% be issued if an attempt is made to load fixltx2e.sty as it is no longer
% needed.
% The latest version and documentation can be found at:
% http://www.ctan.org/pkg/fixltx2e


%\usepackage{stfloats}
% stfloats.sty was written by Sigitas Tolusis. This package gives LaTeX2e
% the ability to do double column floats at the bottom of the page as well
% as the top. (e.g., "\begin{figure*}[!b]" is not normally possible in
% LaTeX2e). It also provides a command:
%\fnbelowfloat
% to enable the placement of footnotes below bottom floats (the standard
% LaTeX2e kernel puts them above bottom floats). This is an invasive package
% which rewrites many portions of the LaTeX2e float routines. It may not work
% with other packages that modify the LaTeX2e float routines. The latest
% version and documentation can be obtained at:
% http://www.ctan.org/pkg/stfloats
% Do not use the stfloats baselinefloat ability as the IEEE does not allow
% \baselineskip to stretch. Authors submitting work to the IEEE should note
% that the IEEE rarely uses double column equations and that authors should try
% to avoid such use. Do not be tempted to use the cuted.sty or midfloat.sty
% packages (also by Sigitas Tolusis) as the IEEE does not format its papers in
% such ways.
% Do not attempt to use stfloats with fixltx2e as they are incompatible.
% Instead, use Morten Hogholm'a dblfloatfix which combines the features
% of both fixltx2e and stfloats:
%
% \usepackage{dblfloatfix}
% The latest version can be found at:
% http://www.ctan.org/pkg/dblfloatfix




%\ifCLASSOPTIONcaptionsoff
%  \usepackage[nomarkers]{endfloat}
% \let\MYoriglatexcaption\caption
% \renewcommand{\caption}[2][\relax]{\MYoriglatexcaption[#2]{#2}}
%\fi
% endfloat.sty was written by James Darrell McCauley, Jeff Goldberg and 
% Axel Sommerfeldt. This package may be useful when used in conjunction with 
% IEEEtran.cls'  captionsoff option. Some IEEE journals/societies require that
% submissions have lists of figures/tables at the end of the paper and that
% figures/tables without any captions are placed on a page by themselves at
% the end of the document. If needed, the draftcls IEEEtran class option or
% \CLASSINPUTbaselinestretch interface can be used to increase the line
% spacing as well. Be sure and use the nomarkers option of endfloat to
% prevent endfloat from "marking" where the figures would have been placed
% in the text. The two hack lines of code above are a slight modification of
% that suggested by in the endfloat docs (section 8.4.1) to ensure that
% the full captions always appear in the list of figures/tables - even if
% the user used the short optional argument of \caption[]{}.
% IEEE papers do not typically make use of \caption[]'s optional argument,
% so this should not be an issue. A similar trick can be used to disable
% captions of packages such as subfig.sty that lack options to turn off
% the subcaptions:
% For subfig.sty:
% \let\MYorigsubfloat\subfloat
% \renewcommand{\subfloat}[2][\relax]{\MYorigsubfloat[]{#2}}
% However, the above trick will not work if both optional arguments of
% the \subfloat command are used. Furthermore, there needs to be a
% description of each subfigure *somewhere* and endfloat does not add
% subfigure captions to its list of figures. Thus, the best approach is to
% avoid the use of subfigure captions (many IEEE journals avoid them anyway)
% and instead reference/explain all the subfigures within the main caption.
% The latest version of endfloat.sty and its documentation can obtained at:
% http://www.ctan.org/pkg/endfloat
%
% The IEEEtran \ifCLASSOPTIONcaptionsoff conditional can also be used
% later in the document, say, to conditionally put the References on a 
% page by themselves.




% *** PDF, URL AND HYPERLINK PACKAGES ***
%
%\usepackage{url}
% url.sty was written by Donald Arseneau. It provides better support for
% handling and breaking URLs. url.sty is already installed on most LaTeX
% systems. The latest version and documentation can be obtained at:
% http://www.ctan.org/pkg/url
% Basically, \url{my_url_here}.




% *** Do not adjust lengths that control margins, column widths, etc. ***
% *** Do not use packages that alter fonts (such as pslatex).         ***
% There should be no need to do such things with IEEEtran.cls V1.6 and later.
% (Unless specifically asked to do so by the journal or conference you plan
% to submit to, of course. )


% correct bad hyphenation here
\hyphenation{op-tical net-works semi-conduc-tor}

% Acronyms Section
\usepackage[acronym]{glossaries}
\newacronym{ble}{BLE}{Bluetooth Low Energy}
\newacronym{iot}{IoT}{Internet of Things}


\begin{document}
%
% paper title
% Titles are generally capitalized except for words such as a, an, and, as,
% at, but, by, for, in, nor, of, on, or, the, to and up, which are usually
% not capitalized unless they are the first or last word of the title.
% Linebreaks \\ can be used within to get better formatting as desired.
% Do not put math or special symbols in the title.
\title{\gls{iot} Security: A Case Study in Reverse Engineering}
%
%
% author names and IEEE memberships
% note positions of commas and nonbreaking spaces ( ~ ) LaTeX will not break
% a structure at a ~ so this keeps an author's name from being broken across
% two lines.
% use \thanks{} to gain access to the first footnote area
% a separate \thanks must be used for each paragraph as LaTeX2e's \thanks
% was not built to handle multiple paragraphs
%

\author{Brandon Kamaka,~\IEEEmembership{Air Force Institute of Technology}
        Nathan Flack,~\IEEEmembership{Air Force Institute of Technology}}% <-this % stops a space


% note the % following the last \IEEEmembership and also \thanks - 
% these prevent an unwanted space from occurring between the last author name
% and the end of the author line. i.e., if you had this:
% 
% \author{....lastname \thanks{...} \thanks{...} }
%                     ^------------^------------^----Do not want these spaces!
%
% a space would be appended to the last name and could cause every name on that
% line to be shifted left slightly. This is one of those "LaTeX things". For
% instance, "\textbf{A} \textbf{B}" will typeset as "A B" not "AB". To get
% "AB" then you have to do: "\textbf{A}\textbf{B}"
% \thanks is no different in this regard, so shield the last } of each \thanks
% that ends a line with a % and do not let a space in before the next \thanks.
% Spaces after \IEEEmembership other than the last one are OK (and needed) as
% you are supposed to have spaces between the names. For what it is worth,
% this is a minor point as most people would not even notice if the said evil
% space somehow managed to creep in.



% The paper headers
\markboth{CSCE 660}%
{Shell \MakeLowercase{\textit{et al.}}: Bare Demo of IEEEtran.cls for IEEE Journals}
% The only time the second header will appear is for the odd numbered pages
% after the title page when using the twoside option.
% 
% *** Note that you probably will NOT want to include the author's ***
% *** name in the headers of peer review papers.                   ***
% You can use \ifCLASSOPTIONpeerreview for conditional compilation here if
% you desire.




% If you want to put a publisher's ID mark on the page you can do it like
% this:
%\IEEEpubid{0000--0000/00\$00.00~\copyright~2015 IEEE}
% Remember, if you use this you must call \IEEEpubidadjcol in the second
% column for its text to clear the IEEEpubid mark.



% use for special paper notices
%\IEEEspecialpapernotice{(Invited Paper)}




% make the title area
\maketitle

% As a general rule, do not put math, special symbols or citations
% in the abstract or keywords.
\begin{abstract}
The industry of embedded devices, referred to as the \gls{iot} is booming. \gls{iot} Analytics estimates the number of \gls{iot} devices in use last year to be more than 7 billion, and the global \gls{iot} Market surpassed \$150 billon in the same year (Lueth, 2018). These embedded systems and their control devices allow humans to more conveniently perform tasks such as view camera feeds, remote-start vehicles, control smart homes and city infrastructure, and communicate with wearables such as smartglasses and implanted medical devices. \gls{iot} devices not only sense the world around them and connect people and systems, but are increasingly used to control the physical world. As this trend continues, security breaches will have greater and greater consequences -- making device and system security ever more important. These \gls{iot} systems of systems include many components such as mobile applications, communication protocols like Bluetooth and Zigbee (Li, Xu, \& Zhao, 2015), and hardware sensors and actuators. This paper examines one such \gls{iot} system, analyzing a door lock activated via fingerprint scanner, mobile application, or physical key. The ZKTeco PL10B Electronic Smart Lock (ZKTECO CO., LTD., 2018) is designed for residential, exterior use. The door lock saves authenticated users’ fingerprint or smart device for access control and authentication. In addition to access control, the provided mobile applications for both Android and iOS operating systems, named ZKBioBL, also provide user permissioning, management, and auditing services (ZKTECO CO., LTD., 2018). This paper analyzes this system end-to-end to determine possible threat vectors and potential vulnerabilities. This includes an examination the software, hardware, firmware (if accessible), fingerprint scanner, physical lock, and the interaction of all of these elements. The physical analysis includes potential physical attacks to compromise the lock, such as lock picking and a magnet attack. The hardware analysis examines the circuitry and embedded control logic and firmware. The software analysis includes reverse engineering the Android mobile app, and finally, the Bluetooth analysis will capture and analyze the two-way communication between the lock and controller. In all of these areas we will look for potential vulnerabilities and attack vectors, especially those that may be applicable in other \gls{iot} devices and systems. The analysis will seek to follow a standardized process to examine the security of other \gls{iot} devices and systems and inform developers and users in order to mitigate current and future vulnerabilities.
\end{abstract}

% Note that keywords are not normally used for peerreview papers.
\begin{IEEEkeywords}
\gls{iot}, \gls{iot}, smart lock, ZKTeco, Bluetooth
\end{IEEEkeywords}



% For peer review papers, you can put extra information on the cover
% page as needed:
% \ifCLASSOPTIONpeerreview
% \begin{center} \bfseries EDICS Category: 3-BBND \end{center}
% \fi
%
% For peerreview papers, this IEEEtran command inserts a page break and
% creates the second title. It will be ignored for other modes.
\IEEEpeerreviewmaketitle



\section{Introduction}
% The very first letter is a 2 line initial drop letter followed
% by the rest of the first word in caps.
% 
% form to use if the first word consists of a single letter:
% \IEEEPARstart{A}{demo} file is ....
% 
% form to use if you need the single drop letter followed by
% normal text (unknown if ever used by the IEEE):
% \IEEEPARstart{A}{}demo file is ....
% 
% Some journals put the first two words in caps:
% \IEEEPARstart{T}{his demo} file is ....
% 
% Here we have the typical use of a "T" for an initial drop letter
% and "HIS" in caps to complete the first word.
\IEEEPARstart{T}{he} industry of embedded devices, commonly referred to as the \gls{iot} is booming. \gls{iot} Analytics estimates the number of \gls{iot} devices in use in 2018 to be more than 7 billion, and the global \gls{iot} Market surpassed \$150 billion in the same year \cite{Scully2017}. This is marked by the connection of everyday appliances and devices together to collect data and provide convenience to users. As more devices are connected together and to the Internet the attack surface is increases and the impact and cost of security breaches increase.  The overall market for \gls{iot} security was estimated to \$703 million in 2017 and is expected to grow to \$4.3 billion in 2022 \cite{Scully2017}. This rise follows the growing trend in global \gls{iot} technology spending, which is predicted by Forbes Magazine to reach \$1.2 trillion in 2022 \cite{Columbus2018}.

\bigskip

The \gls{iot} refers to networks of physical objects represented and accessed in and from the virtual world. In an \gls{iot}, devices are programmed to exchange and process data based on the needs of the consumer or developer \cite{Li2015}. The end goal of \gls{iot} is to provide humans connectivity and intelligence they desire in daily life whether on a small scale in a smart home or on a large scale in a global manufacturing company \cite{Li2015}. 

\bigskip

Many sectors have contributed to this increase including business, social networks, healthcare, infrastructure, and security and surveillance \cite{Li2015}. Security include the military, which is embracing and benefiting from this technology wave. \gls{iot} promises a more connected battlefield where information travels quickly and artificial intelligence and machine learning can assist human decision-makers. "\gls{iot}-enabled technologies are expected to provide decision-makers with real-time / near-real-time information about the status of assets." As most applicable to this case study, these devices may interact with mobile applications and communicate via Bluetooth \cite{Miller}. Given the size of the market outside of the defense community it is expected that many of the same vulnerabilities and security flaws that exist within commercial products will be present in military systems. 

\bigskip

Several models have emerged defining the basic architecture of an \gls{iot} system. Li, Xu, and Zhao define the layers as sensing, network, service, and interface \cite{Li2015}. Scully and Van Aken define a similar structure of devices, communication, cloud, and application \cite{Scully 2017}.  

\bigskip

This case study examines an \gls{iot} system that, if leveraging all capabilities, does not use the cloud, but instead stores data on the device and uses Bluetooth to communicate with a mobile management application.

\bigskip

Smart Locks for home use have been the topic of several recent \gls{iot} security analyses and reverse engineering projects. These devices in particular draw attention because they seem to provide better tracking and home security, but they also open up additional attack vectors for burglars and others who want to cuase physical, financial, and psychological harm \cite{Fernandes2016}. Fernandes, Jung, and Prakash examined door locks as part of a survey of smart devices in 2015 and were able to steal lock pin-codes and implant malicious lock codes to unlock the smart lock physically or remotely at the time of the malicious actors choosing \cite{Fernandes2016}. In one case study, Symantec found 10 vulnerabilities associated with network traffic from various devices, including smart locks. The communication between the cloud application and the devices were vulnerable to cross-site scripting, path traversal, unrestricted file uploading allowing remote code execution and SQL injection attacks. These could have allowed anyone to remotely open smart locks \cite{Barcena2015}.

% B. Here's some similar work on locks and security in the \gls{iot} context
% C. Here's our lock -
%     1. Basic description (model, capabilities, manufacturer, etc ...)



\section{Methodology}

Although emphasis was placed on reverse engineering the software components of the ZKTEco PL-10B (ZK lock), security cannot be assessed in isolation [CITATION].  As such, the chosen methodology involved a holistic view of both hardware and software components.  In particular, The first goal was to obtain the manufacturer's firmware in order to assess how packets were created, shaped, and consumed by the controller and host devices.  Unfortunately, obtaining the firmware was not achievable within the confines of this effort, which significantly hampered some elements of the reverse engineering attempt.

\subsection{Hardware}
% What we will/have looked at:
%  Phone details?
% PCB Layout
%         2. JTAG, UART (?), USB/Serial(?) connections
%         3. Removing conformal coating



\subsection{Software}
% What we will/have looked at:
%         1. App behavior -
%             a. decompiling and examining using jadx
%             b. Examining locally-stored data
%             c. Permissions
%             d. Default to static super user password (pairing and admin passwords)


The ZK lock is designed to operate in conjunction with an iOS or Android application that provides authentication and command and control capabilities to the lock mechanism.  As such, it is integral to this effort to understand the behavior of the application.  The testing used two "rooted" Google Pixel smartphones, one "rooted" Google Pixel 2 XL smartphone, and an iPhone [IPHONE VERSION NAME] as the client controller devices.  Prior to beginning testing operations, each controller had the appropriate version (iOS/Android) of the ZK Bio BL [VERSION NUMBERS] installed.  In the software auditing context, two logical sets of components were identified: the behavior of the mobile app on the controller device, and the communication between the controller and the Bluetooth host device (the ZK lock) which implements \gls{ble}.

\bigskip %add a newline to separate paragraphs
\subsubsection{App Behavior}

Before initiating communications analysis, the general behavior of the Android application was explored.  The audit made use of Android Studio [VERSION NUMBER] to decompress the ZK Bio BL APK and examine its manifest file for permissions information, activities and any associated filters, and notable resources (e.g. pinned certificates).  Additionally, the Android Debug Bridge (ADB) software was used on the rooted controller devices to examine application-specific files stored on the device.  The final tool used to assist in the exploration of ZK Bio BL app behavior was jadx, which provides Java decompilation capabilities [CITATION] and can be used to view replicated native code to provide insight into the app's internal behavior.  The application was also used according to the manufacturer's instructions to understand its normal modes of operation.

\bigskip
\subsubsection{Bluetooth Communication}
%         2. Bluetooth communication
%             a. Capturing first party data
%             b. Capturing third party data
%             c. attempting to understand the handshake
%             d. Attempting to fake the handshake
%             e. Attempting to understand the commands
%             f. Attempting to fake the commands

The communication analysis methodology involved the creation of multiple test cases covering a range of lock and application behaviors and states.  The ZK Bio BL application recognizes 4 distinct user/device types: administrator, user, temporary user, and unauthorized user [CHECK ACTUAL ROLE NAMES].  Each type of user except unauthorized is permitted to send an unlock command to the lock.  Additionally, the system permits any type of user to elevate privileges using a superuser-style passcode.  Test cases were prepared for each user role to initiate initial device pairing with the bluetooth host, and to elevate privileges into the superuser role.  In addition to using the manufacturer's application to perform communication functions, use was also made of nRF Connect.  This application permits controller devices to audit bluetooth advertisements, examine services and profiles offered by host devices, and to read and write arbitrary data to characteristic values within advertised attributes that possess readable and writeable properties.  Test cases were prepared whereby we attempted to use nRF connect to mimic ZK Bio BL behavior by sending crafted characteristic value reads and writes [nRF Functionality CITATION].

\bigskip

In all test cases, the communication was recorded by enabling the BTSnoop logging capabilities provided by the Android operating system [BTSnoop log capabilities CITATION].  After the required actions were performed on the system, the log was retrieved from the controller device and analyzed with Wireshark - a packet capture and analysis application provided open source [Wireshark capabilities CITATION].  Where possible, distinct actions and roles were captured as separate logs.


\section{Implementation}

Also known as "findings"

\subsection{Hardware}
No content

\subsection{Software}
        % 1. App
        %     a.We decompiled.  It's obfuscated and our failure to dump the firmware made this impossible to disentangle
        %     b. No relevant data appears to be 
        %     c. Insecurity in default passwords not automatically changed. Should be unique from the factory.

\bigskip        
\subsubsection{App Behavior}

Application behavior in normal use indicated the developers used an idiosyncratic security paradigm for their multi-user lock system.  In particular, the application does not permit any distinction between users and their devices.  User accounts are created only when a new controller device with ZK Bio BL pairs with the ZK lock.  On pairing, an \textit{unauthorized user} data object is created in the ZK lock's memory and on the controller device.  This data object has fields for [INSERT ZK BIO BL DATA FIELDS FOR USER ACCOUNTS].  There is no method to associate an existing user account with a new device, nor an existing device with a new user account unless a system user uninstalls the application and re-installs it -- essentially recreating a "new" device.  Because of this behavior, it is neither possible nor useful to distinguish "user" from "device" accounts.

\bigskip

The application provides no means of remotely editing accounts.  That is, account metadata and permissions can only be changed on the specific device associated with that account. As such, all unauthorized users remain in the unauthorized role until the device is physically accessible by a system administrator in possession of the user-configurable superuser passkey.  Once such a system administrator enters the superuser key into the application on the local controller device associated with an account, then that device's account information -- including the user role associated with that account -- can be permanently edited.  On application close, and \textit{only on application close}, the local account loses superuser privileges and reverts to its associated user role (possibly a new role, if one was set during the superuser phase).

\bigskip

Although there are four, distinct semantic labels associated with the account roles, in practice there are only two, functional roles;  Administrators, users, and temporary users can unlock the ZK lock while unauthorized users cannot.  Although marketing information purports that device administrators can perform other, standard administrative functions -- e.g. account creation, per-account privilege scheduling, and security auditing -- these capabilities are either not present or, in the case of account creation, are dependant on entering the superuser key into the appropriate controller device and are wholly unassociated with the account information of the system administrator.

\bigskip

[INSERT TABLE OF USER ROLES/PERMISSIONS HERE?]

\bigskip

The application makes use of a standard, fairly limited permissions schema [INSERT ACTUAL PERMISSIONS HERE].  It also stores very little data [INSERT ACTUAL DATA].  This is not surprising given the limited nature of user account management and the application's inability to distinguish users and devices.  

\bigskip

[BAD IMPLEMENTATION VIS-A-VIS DEFAULT PAIRING BEHAVIOR]

\bigskip
\subsubsection{Bluetooth Communication}
%     a. Examination of the handshake
%     b. Key insights into the communication
%         a. The passkey
%         b. the superuser key
%         c. various ATT/GATT handles
%         d. the advertising content filter byte changes
%         e. The bizarre pairing nature

Since the host device firmware was not obtainable, we could not fully assess the handshaking procedure and develop a holistic model of a complete pairing and unlocking procedure on the ZK lock.  However, multiple features were extracted from which we can derive at least some critical behavior.  Additionally, these features provide reasonable starting points for future investigation.

\bigskip

Like many \gls{ble} \gls{iot} devices, the ZK lock does not implement any native \gls{ble} security pairing mechanism [MANY SECURE APPS DON'T USE SECURE STUFF CITATION].  Instead, ZKTeco have elected to permit insecure \gls{ble} connections that then use application logic to authorize command and control from controller devices.  The authorization 

\section{Conclusion}
The conclusion goes here.

\subsection{Primary Findings}

% 1. Reverse engineering (mobile) communication really requires a strong, foundational knowledge of both sides of the communication.  Fundamentally, our inability to gather data about the system architecture and firmware behavior was a killer.
%         2. Regardless of our failure, there are apparent vulnerabilities in the BT implementation.
%             a. Total lack of documentation isn't a good sign: security through obscurity.
%             b. BT isn't implemented well here
%             c. A lot of non-standard behavior that an end-user would never notice
%         3. Conclusion about default admin pairing code and multiple access avenues. Usability still king. 


Our primary findings
Reverse engineering a mobile application requires a strong, foundational knowledge of both sides of the communication. Fundamentally, our inability to gather data about the system architecture and firmware behavior thwarted or restricted our reverse engineering efforts.

Regardless of the many failures, there are apparent vulnerabilities in the Bluetooth implementation. First, a total lack of documentation isn't a strength of the system overall. The lock follows the "security through obscurity" mantra \cite{Mercuri, 2003}. Secondly, the Bluetooth communication is not implemented well....
Last, there is significant non-standard behavior that an end-user would not notice, but is detrimental to their security. 
The third significant security flaw is the use of a default administrator password for all devices along with multiple independent avenues of access. If a user only uses the fingerprint reader on the 
Breaking into this lock means that theAllowing access to 


\subsection{Future Work}

        1. Frida hooking
        2. HW-level analysis of debugging interfaces
        3. Android application to more fully control BT behavior.

% An example of a floating figure using the graphicx package.
% Note that \label must occur AFTER (or within) \caption.
% For figures, \caption should occur after the \includegraphics.
% Note that IEEEtran v1.7 and later has special internal code that
% is designed to preserve the operation of \label within \caption
% even when the captionsoff option is in effect. However, because
% of issues like this, it may be the safest practice to put all your
% \label just after \caption rather than within \caption{}.
%
% Reminder: the "draftcls" or "draftclsnofoot", not "draft", class
% option should be used if it is desired that the figures are to be
% displayed while in draft mode.
%
%\begin{figure}[!t]
%\centering
%\includegraphics[width=2.5in]{myfigure}
% where an .eps filename suffix will be assumed under latex, 
% and a .pdf suffix will be assumed for pdflatex; or what has been declared
% via \DeclareGraphicsExtensions.
%\caption{Simulation results for the network.}
%\label{fig_sim}
%\end{figure}

% Note that the IEEE typically puts floats only at the top, even when this
% results in a large percentage of a column being occupied by floats.


% An example of a double column floating figure using two subfigures.
% (The subfig.sty package must be loaded for this to work.)
% The subfigure \label commands are set within each subfloat command,
% and the \label for the overall figure must come after \caption.
% \hfil is used as a separator to get equal spacing.
% Watch out that the combined width of all the subfigures on a 
% line do not exceed the text width or a line break will occur.
%
%\begin{figure*}[!t]
%\centering
%\subfloat[Case I]{\includegraphics[width=2.5in]{box}%
%\label{fig_first_case}}
%\hfil
%\subfloat[Case II]{\includegraphics[width=2.5in]{box}%
%\label{fig_second_case}}
%\caption{Simulation results for the network.}
%\label{fig_sim}
%\end{figure*}
%
% Note that often IEEE papers with subfigures do not employ subfigure
% captions (using the optional argument to \subfloat[]), but instead will
% reference/describe all of them (a), (b), etc., within the main caption.
% Be aware that for subfig.sty to generate the (a), (b), etc., subfigure
% labels, the optional argument to \subfloat must be present. If a
% subcaption is not desired, just leave its contents blank,
% e.g., \subfloat[].


% An example of a floating table. Note that, for IEEE style tables, the
% \caption command should come BEFORE the table and, given that table
% captions serve much like titles, are usually capitalized except for words
% such as a, an, and, as, at, but, by, for, in, nor, of, on, or, the, to
% and up, which are usually not capitalized unless they are the first or
% last word of the caption. Table text will default to \footnotesize as
% the IEEE normally uses this smaller font for tables.
% The \label must come after \caption as always.
%
%\begin{table}[!t]
%% increase table row spacing, adjust to taste
%\renewcommand{\arraystretch}{1.3}
% if using array.sty, it might be a good idea to tweak the value of
% \extrarowheight as needed to properly center the text within the cells
%\caption{An Example of a Table}
%\label{table_example}
%\centering
%% Some packages, such as MDW tools, offer better commands for making tables
%% than the plain LaTeX2e tabular which is used here.
%\begin{tabular}{|c||c|}
%\hline
%One & Two\\
%\hline
%Three & Four\\
%\hline
%\end{tabular}
%\end{table}


% Note that the IEEE does not put floats in the very first column
% - or typically anywhere on the first page for that matter. Also,
% in-text middle ("here") positioning is typically not used, but it
% is allowed and encouraged for Computer Society conferences (but
% not Computer Society journals). Most IEEE journals/conferences use
% top floats exclusively. 
% Note that, LaTeX2e, unlike IEEE journals/conferences, places
% footnotes above bottom floats. This can be corrected via the
% \fnbelowfloat command of the stfloats package.





% if have a single appendix:
%\appendix[Proof of the Zonklar Equations]
% or
%\appendix  % for no appendix heading
% do not use \section anymore after \appendix, only \section*
% is possibly needed

% use appendices with more than one appendix
% then use \section to start each appendix
% you must declare a \section before using any
% \subsection or using \label (\appendices by itself
% starts a section numbered zero.)
%



% use section* for acknowledgment
% \section*{Acknowledgment}


% The authors would like to thank...


% Can use something like this to put references on a page
% by themselves when using endfloat and the captionsoff option.
\ifCLASSOPTIONcaptionsoff
  \newpage
\fi



% trigger a \newpage just before the given reference
% number - used to balance the columns on the last page
% adjust value as needed - may need to be readjusted if
% the document is modified later
%\IEEEtriggeratref{8}
% The "triggered" command can be changed if desired:
%\IEEEtriggercmd{\enlargethispage{-5in}}

% references section

% can use a bibliography generated by BibTeX as a .bbl file
% BibTeX documentation can be easily obtained at:
% http://mirror.ctan.org/biblio/bibtex/contrib/doc/
% The IEEEtran BibTeX style support page is at:
% http://www.michaelshell.org/tex/ieeetran/bibtex/
%\bibliographystyle{IEEEtran}
% argument is your BibTeX string definitions and bibliography database(s)
%\bibliography{IEEEabrv,../bib/paper}
%
% <OR> manually copy in the resultant .bbl file
% set second argument of \begin to the number of references
% (used to reserve space for the reference number labels box)
\begin{thebibliography}{1}

\bibitem{IEEEhowto:kopka}
H.~Kopka and P.~W. Daly, \emph{A Guide to \LaTeX}, 3rd~ed.\hskip 1em plus
  0.5em minus 0.4em\relax Harlow, England: Addison-Wesley, 1999.

\end{thebibliography}

% biography section
% 
% If you have an EPS/PDF photo (graphicx package needed) extra braces are
% needed around the contents of the optional argument to biography to prevent
% the LaTeX parser from getting confused when it sees the complicated
% \includegraphics command within an optional argument. (You could create
% your own custom macro containing the \includegraphics command to make things
% simpler here.)
%\begin{IEEEbiography}[{\includegraphics[width=1in,height=1.25in,clip,keepaspectratio]{mshell}}]{Michael Shell}
% or if you just want to reserve a space for a photo:


% You can push biographies down or up by placing
% a \vfill before or after them. The appropriate
% use of \vfill depends on what kind of text is
% on the last page and whether or not the columns
% are being equalized.

%\vfill

% Can be used to pull up biographies so that the bottom of the last one
% is flush with the other column.
%\enlargethispage{-5in}


% that's all folks
\end{document}


