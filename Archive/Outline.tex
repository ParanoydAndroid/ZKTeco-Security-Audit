This is just here to write outline ideas, not for compiling into the main document.

Abstract

I. Introduction
    A. IoT devices are important and there are bunches of them
    B. Here's some similar work on locks and security in the IoT context
    C. Here's our lock -
        1. Basic description (model, capabilities, manufacturer, etc ...)
II. Methodology - We intend to perform a security audit broken up into two, major categories: HW and SW
    A. Hardware - What we will/have looked at:
        1. PCB Layout
        2. JTAG, UART (?), USB/Serial(?) connections
        3. Removing conformal coating
    B. Software - What we will/have looked at:
        1. App behavior -
            a. decompiling and examining using jadx
            b. Examining locally-stored data
            c. Permissions
            d. Default to static super user password (pairing and admin passwords)
        2. Bluetooth communication
            a. Capturing first party data
            b. Capturing third party data
            c. attempting to understand the handshake
            d. Attempting to fake the handshake
            e. Attempting to understand the commands
            f. Attempting to fake the commands
III. Implementation (findings): failure all around
    A. HW:
    B. SW:
        1. App
            a.We decompiled.  It's obfuscated and our failure to dump the firmware made this impossible to disentangle
            b. No relevant data appears to be 
            c. Insecurity in default passwords not automatically changed. Should be unique from the factory.
        2. BT
            a. Examination of the handshake
            b. Key insights into the communication
                a. The passkey
                b. the superuser key
                c. various ATT/GATT handles
                d. the advertising content filter byte changes
                e. The bizarre pairing nature
    C. Implementation
        
IV. Conclusion - our primary findings and future work
    A. primary conclusions
        1. Reverse engineering (mobile) communication really requires a strong, foundational knowledge of both sides of the communication.  Fundamentally, our inability to gather data about the system architecture and firmware behavior was a killer.
        2. Regardless of our failure, there are apparent vulnerabilities in the BT implementation.
            a. Total lack of documentation isn't a good sign: security through obscurity.
            b. BT isn't implemented well here
            c. A lot of non-standard behavior that an end-user would never notice
        3. Conclusion about default admin pairing code and multiple access avenues. Usability still king. Applicable to other IoT devices, etc.
    B. Future directions
        1. Frida hooking
        2. HW-level analysis of debugging interfaces
        3. Android application to more fully control BT behavior.